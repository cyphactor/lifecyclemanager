\documentclass{article}
\usepackage{lm-latex}
\usepackage{epic,eepic,xcolor,rotating}

\title{Metrics}

\begin{document}

\maketitle

\tableofcontents


\section{Project Abstract}

Lifecycle Manager is a full-featured collaborative project management
environment seamlessly integrating project milestones, requirements, tickets,
documentation, and artifact versioning in a multi-user concurrent
service-oriented infrastructure. Lifecycle Manager may be used to record,
predict, and evaluate performance of project features and teams, allowing
better planning and management of development efforts ranging from simple to complex.




\section{Representation schemata}

Explain reasons for vector, and what it really is represented.

\subsection{Requirement vectors}

Requirements are made up of several components. For any functional
requirement at some time $t$, we define the vector $\vec{r}_t$ as
\begin{align}
\label{eq:def_of_rt}
\vec{r}_t &=& <\comp,\fp,\obj,\status,\desc,\user,\ctime>,\\
\text{where} && \begin{cases}
  \comp     & \text{is the component}\\
  \fp       & \text{is the functional primitive}\\
  \obj      & \text{is the object}\\
  \status   & \text{is `enabled' or `disabled'}\\
  \desc     & \text{is the description}\\
  \user     & \text{is the user responsible for the current state}\\
  \ctime    & \text{is the time of the last change}
  \end{cases} \notag
\end{align}
The component $\desc$ is underlined because it may contain requirement
citations.

Because $\vec{r}_t$ shows the requirement as a snapshot in time, we
can represent a requirement's evolution, or sequence of states, as
\begin{equation}
\label{eq:def_of_r}
\seq{r} = (\vec{r}_{t_0}, \vec{r}_{t_1}, \dots).
\end{equation}
The only redundancy in $\seq{r}$ is in the $\comp$, $\fp$, and $\obj$
components, as these three fields do not change. Functional requirements
are uniquely identified by these first three fields in $\vec{r}_t$.


\subsection{Component, functional primitive, and object dictionaries}


\subsection{Predicates}
\label{sec:predicates}

In order to restrict metrics to a certain subset of requirements of
interest, we utilize a predicate function that discriminates among all
states of all known requirements. For example, such a predicate
function may select the most recent states of requirements that were
created by a specific user, or requirements which have been cited in
defect tickets. Predicate functions are indicated by the symbol
$\pred{p}$. Again, predicates allow the same metric to be applied in a
range of contexts by selectively including specific sets of
requirements.


\subsection{Requirement matrix}

Most of the metrics that follow, however, do not operate on all of the
information stored in a $\vec{r}_t$ vector. Only the $\fp$ and $\obj$
fields are considered for the moment. We construct a matrix of
functional primitive-object pairings
\begin{equation}
\mat{R}_t[\pred{p}] = \begin{tabular}{c|ccc}
  & $\obj_1$ & $\obj_2$ & $\cdots$\\
  \hline
  $\fp_1$ & $c_{1,1}$ & $c_{1,2}$ & $\cdots$ \\
  $\fp_2$ & $c_{2,1}$ & $c_{2,2}$ & $\cdots$ \\
  $\vdots$ & $\vdots$ & $\vdots$ & $\ddots$
\end{tabular}
\end{equation}
where $c_{i,j}$ is the count of requirements $\seq{r}$ with property
$\pred{p}$ whose vectors $\vec{r}_t$ have second and third components
$\fp$ and $\obj$, respectively. Note that $\mat{R}_t$ will likely not
be square, since there's no reason to believe that, in general, a
requirements specification will employ sets of functional
primitives and objects of the same size.

The matrix $\mat{R}_t$ represents a snapshot at time $t$. The counts
of requirements with property $\pred{p}$ will certainly change over
time as requirements are created and disabled. Such a matrix
representation allows us to easily model these changes.

In the subsequent metrics, the following notations will be used in
reference to the matrix $\mat{R}_t[\pred{p}]$ for some known
$\pred{p}$ and time $t$. Let $M_f$ be the row-sum of
$\mat{R}_t[\pred{p}]$ of the row representing the functional primitive
$f$; $M_b$ be the column-sum of $\mat{R}_t[\pred{p}]$ of the column
representing the object $b$; and $M_{f,b}$ be the value of
$\mat{R}_t[\pred{p}]$ at row $f$, column $b$.

With the above notations, it follows that the total count of
requirements with property $\pred{p}$ at time $t$ is equivalently
$\sum_{f \in F} M_f$, $\sum_{b \in B} M_b$, and $\sum_{f \in F}
\sum_{b \in B} M_{f,b}$. We denote this count as simply $M$.


\subsection{Other design artifacts}

Many artifacts are created during the design process. Besides
requirements, Lifecycle Manager stores wiki content, tickets (defects,
enhancements, tasks, etc.), and source code, with support for
additional artifacts via plugins. The essential property Lifecycle
Manager needs in a design artifact is a method of citing
requirements. In wiki content, requirements may be cited anywhere. In
a ticket, requirements may be cited in the description or comment
field. Source code and repository commit messages may also cite
requirements. If a design artifact (such as an image) is unable to
cite requirements in plain text, Lifecycle Manager is unable to
consider any relations between the artifact and requirements.

In order to limit a metric of interest to requirements that have been
cited in some particular category of design artifacts, an appropriate
predicate may be used (see Section \ref{sec:predicates}).


\section{Information theory overview}

Besides counting metrics, which are simply summations and percents of
wholes, Lifecycle Manager uses information theoretical metrics in
order to gain more insight than counting metrics provide. Such insight
relates to the significance of certain phenomenon, in terms of how
much information is conveyed.

\subsection{Information entropy}

Information theory was developed as a means of measuring the amount of
information carried in a physical signal or channel. Shannon
\cite{shannon} quantified the amount of ``surprise'' from a received
symbol in a stream of symbols. If the reception of a symbol is very
unlikely, i.e. surprising, then it carries a high amount of
information. The total information content in a channel is its
``information entropy'' (due to the mathematical similarities with
Boltzmann's equations and thermodynamic entropy \cite{chat_steph}).

As an example, consider the symbols of the Latin alphabet and a
transmission of an English sentence. Each letter has a certain
probability of occurring, derived from the frequencies in a wide
collection of English texts. The letter `e' is common, while `x' is
uncommon. Therefore, when receiving the transmission, i.e. the
sentence, one letter at a time, there will be little surprise if `e'
is transmitted and much surprise if `x' is transmitted. So `x' carries
more information than `e'. The entropy of the transmission is the
total information carried by the transmission of each individual
symbol. A transmission with many `e's will have less information
entropy than a transmission with many `x's, since each `x' carries
more information than each `e' \cite{shannon}.

Shannon \cite{shannon} quantified this element of ``surprise'' in the
following form:
\begin{equation}
  \label{eq:I(s)}
  I_s = - \log_2 P(s).
\end{equation}
In this equation, $I_s$ is the information content of a particular
symbol $s$ from a set of symbols $S$, and $P(s)$ is the likelihood, or
probability, of the symbol occurring. Particularly, $\sum_{s \in S}
P(s) = 1$. The equation shows that if a symbol has a decreased
likelihood of occurring, its information content increases. If a
symbol's likelihood of occurring reduces by half, one more unit of
information is conveyed if it does occur.

The average information content of the transmission is the weighted
sum of the information content of each symbol (also known as the
expected value):
\begin{equation}
  \label{eq:I_avg}
  I_{\textrm{avg}} = - \sum_{s \in S} P(s) \log_2 P(s).
\end{equation}
Note that $I$ is maximized when the probably $P(s)$ of each symbol
occurring is equal.

These definitions of information entropy are the foundation of our
information entropy metric.

\subsection{Information entropy metric}

We utilize information entropy to determine the significance, in terms
of information content, of specific functional requirements in a
requirements document written using the Common Functional Modeling
Framework. Previous work \cite{abd-el-hafiz} has shown Shannon's
formulation of information entropy \cite{shannon}, rather than other,
more generalized forms \cite{aczel_daroczy}, is most meaningful when
measuring software systems. Since we position our information entropy
metric as a requirements metric, the entities we are measuring are
functional requirements. Each functional requirement has a measurable
amount of information content, as do the components composing these
functional requirements.

Other uses of information entropy in requirements metrics often focus
on metrics of software complexity and predictions of software
defects. The work of Khoshgoftaar and Allen \cite{khosh_allen}
analyzed software written in assembly language. The average
information content of a program was calculated using the set of
assembly language operators and the frequency of their use. In program
code with greater average information content, which occurs when
operators are used more equally, defects were more common. This
enforced the authors' belief that if more operators are used there
will be more faults, as programmers are only well-experienced with a
proper subset of all available operators.

Information entropy is also used to measure the complexity of
inter-relations in an object-oriented design: classes that call
methods of other classes \cite{chat_steph}. The authors argue that
adding new functionality significantly raises average information
content, therefore affecting a large number of existing classes (that
is, added functionality calls more methods of existing classes). This
is an effect a developer would hope to reduce since it adds to
software complexity.

Further uses of information entropy in software measurement are
detailed by Abd-el-Hafiz \cite{abd-el-hafiz} and Etzkorn and Gholston
\cite{etzkorn_gholston}.

The Shannon entropy measure has eight properties that make it useful
for measuring the information content of software artifacts
\cite{abd-el-hafiz}. These properties show the consistency of our
metric with classical interpretations of information entropy:
\begin{enumerate}
\item \textit{Nonnegative:} All functional requirements have
  nonnegative information content.
\item \textit{Symmetric:} The order in which functional requirements
  are written is not important.
\item \textit{Normal:} If a functional requirement is composed of a
  functional primitive (or object) that operates on (or is operated on
  by) only two distinct objects (or functional primitives) that
  equally occur in the requirements document, than that functional
  primitive (or object) contains 1 unit of information.
\item \textit{Expansible:} Functional primitives (or objects) do not
  gain information from the objects (or functional primitives) that
  they do not operate on (or are operated on by).
\item \textit{Decisive:} If both the functional primitive and object
  of a functional requirement are unique in the requirements document,
  then the functional requirement contains no information.
\item \textit{Additive:} The information content in any two functional
  requirements is the sum of the information content of each. This is
  the justification for the formulation of the information content of
  a component, $I(c)$.
\item \textit{Subadditive:} The information content of components is
  not greater than the sum of the information content of all included
  functional requirements.
\item \textit{Maximality:} The information content of a functional
  requirement is maximized when the frequency of use of all functional
  primitives and objects relating to the functional requirement are
  equal (that is, when $P_f(b)$ and $P_b(f)$ are equal for all
  functional primitives $f$ and objects $b$).
\end{enumerate}


\section{Static metrics}

\subsection{Pointwise mutual information}

A mutual information metric indicates the strength of a pairing of two
random variables in an information source. In our formulation, the
random variables are which functional primitive and which object are
used in functional requirements. That is to say, we measure the
strength of pairing functional primitives with objects in a set of
requirements.

For the pointwise mutual information metric, we look at individual
functional primitives and objects.

\begin{equation}
\Pmi_R(\pred{p}, \vec{r}_t) \equiv \log_2 \frac{\Prj(\pred{p}, \vec{r}_t)}{\Prn(\pred{p}, f)\Prn(\pred{p}, b)},
\end{equation}
where $\vec{r}_t = <\_,f,b,\_,\_,\_,\_>$ and
\begin{equation*}
\Prj(\pred{p}, \vec{r}_t) = M_{f,b} / M, \quad \Prn(\pred{p}, f) = M_f / M, \quad \Prn(\pred{p}, b) = M_b / M
\end{equation*}
Note that $\Prn(\pred{p}, f)$ and $\Prn(\pred{p}, b)$ represent the
count of requirements at time $t$ with property $\pred{p}$ and
functional primitive $f$ or object $b$, respectively, divided by the
count of \textit{all} requirements at time $t$ with property
$\pred{p}$.

The value $\Pmi_R$ is zero if the functional primitive and object of
the requirement are completely independent of each other. In this
case, $M_f M_b / M = M_{f,b}$.  The functional primitive and object
are paired in the set of requirements less than would be expected from
completely independent pairings, $\Pmi_R < 0$, and when the pairing
occurs more often than chance pairings, $\Pmi_R > 0$. 

\subsection{Mutual information}

The general mutual information metric is the expected value of the
pointwise mutual information metric for any functional primitive and
object. Thus, it is a single value and best measured with respect to
some other changing variable, such as time. We define mutual
information simply as
\begin{equation}
\label{eq:mutual_info}
I_R(\pred{p}, t) = \sum_{f \in F} \sum_{b \in B} \Prj(\pred{p}, \vec{r}_t) \Pmi_R(\pred{p}, \vec{r}_t),
\end{equation}
where $\vec{r}_t$ has second and third components $f$ and $b$,
respectively.


\bibliographystyle{abbrv}
\bibliography{cumulative}


\end{document}
