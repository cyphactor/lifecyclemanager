\documentclass{article}
\usepackage{lm-latex}

\title{Proposed Ontology}

\begin{document}

\maketitle

\tableofcontents

\begin{abstract}
Put abstract here\ldots
\end{abstract}



\section{Project Abstract}

Lifecycle Manager is a full-featured collaborative project management
environment seamlessly integrating project milestones, requirements, tickets,
documentation, and artifact versioning in a multi-user concurrent
service-oriented infrastructure. Lifecycle Manager may be used to record,
predict, and evaluate performance of project features and teams, allowing
better planning and management of development efforts ranging from simple to complex.





\section{Entities}


\subsection{Code}

Code is that entity that must exist for any other entities to have an ontological
context, a reason to exist. Note that \textit{code} does not mean some
particular source code in some particular programming language, as found in some
particular directory like \texttt{src}. Instead, \textit{code} refers to any
final design document or documents whose execution or interpretation (by human
or machine) requires only an automated, scripted transformation into such an
executable or interpretable format that requires
sufficiently-minimal machine resources (sufficiently-minimal means does not
require more design work in order to realize such a transformation).

For example, \LaTeX\ or DocBook format constitutes code for a document, since
transformation into PDF, HTML, etc is sufficiently-minimal. C++ or Java files
are code since transformation into binary or bytecode format is
sufficiently-minimal. And so on.

We take the position that code is the outcome of design, or perhaps the
documentation of design. Projects are measured on the quality of their code.
While services requiring person-effort (such as installation, troubleshooting,
etc) are entities in their own right and do not necessarily involve any creation of
modification of code, Lifecycle Manager tracks the evolution of code and nothing
more. 

\textit{Code} is often also labeled \textit{source code}. In the license of
Lifecycle
Manager, the GNU General Public License version 2, the definition of source is
explained as:
\begin{quote}
The source code for a work means the preferred form of the work for
making modifications to it.
\end{quote}

The definition in this document and in the license are comparable.

\subsection{Tickets}

A task, enhancement, or defect is documented in a \textit{ticket}. Tickets
relate to code in that such tasks, enhancements, and defects relate to code. 

\subsection{Time}

Two entities relate to time: \textit{code} and \textit{tickets}. Changes to code
(`commits') occur at specific times. Tickets progress through their workflow
(creation, modification, resolution, reopening, etc) at specific times.
Individual tickets as well may require resolution before certain dates (`milestones').

\subsection{Requirements}

A project consists of requirements. These requirements are elicited from
stakeholders and define the goals of successive development. In this respect
requirements are like the compass that direct changes in code, the content and
purpose of tickets, and timelines (`milestones').

Requirements evolve as well; requirements are redefined, added, and abandoned.
They are also prioritized. However, they do not hold a fundamental relationship
to time (except in the previously-stated respects) because they are forever
valid (in whatever is each requirement's current form). A requirement is not
`no longer a requirement' after some period of time, nor is it relevant (beyond
the generation of reports) when a requirement was redefined, added, or abandoned.
Once a requirement is stated as such (a `requirement'), it has always been and
will always be a requirement (unless it is redefined or abandoned).

Requirements, being the compass of development, are the overriding context for
any development activity. Every activity occurs because of some requirement (for
if an activity occurred without being required, that activity is simply a waste
of effort and time). Thus, code and tickets will typically specify to which
requirement or requirements they relate, which requirement or requirements they
help \textit{satisfy} or \textit{satisfice}.


\end{document}