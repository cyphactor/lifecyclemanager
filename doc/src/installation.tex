\documentclass{article}
\usepackage{lm-latex}

\title{Installation}

\begin{document}

\maketitle

\tableofcontents


\section{Project Abstract}

Lifecycle Manager is a full-featured collaborative project management
environment seamlessly integrating project milestones, requirements, tickets,
documentation, and artifact versioning in a multi-user concurrent
service-oriented infrastructure. Lifecycle Manager may be used to record,
predict, and evaluate performance of project features and teams, allowing
better planning and management of development efforts ranging from simple to complex.




\section{Software Requirements}

Lifecycle Manager is built as a \href{http://trac.edgewall.org/}{Trac} plugin
and requires no modification to Trac's codebase or an existing Trac installation.
The plugin has been tested only with the latest stable version of Trac: 
\href{http://trac.edgewall.org/wiki/TracDownload}{Trac 10.x}.

In this respect, the other software requirements for using Lifecycle Manager are
only those of Trac. See Trac's
\href{http://trac.edgewall.org/wiki/TracInstall}{Installation Guide} for further
information.

\subsection{Python Version}
\label{sec:python-version}

Of particular interest, however, is your version of Python. Lifecycle Manager is
built to a particular version of Python. Please check your Python documentation
for your current version.


\section{Installation Procedure}

Download Lifecycle Manager from this location: XXX. Be sure to choose the
version matching your Python installation (see section \ref{sec:python-version} above).

The distribution of Lifecycle Manager is a `tarball', or tape-archive file
compressed with GNU's gzip. Download the tarball and extract as follows (where
\texttt{\$} indicates the shell prompt):
\begin{verbatim}
$ tar xzvf TracRequirements-0.1.tar.gz
\end{verbatim}
You will see output similar to the following:
\begin{verbatim}
AUTHORS
INSTALL
NEWS
README
TODO
TracRequirements-0.1-py2.4.egg
pdf/coding_policy.pdf
pdf/database_schema.pdf
pdf/dream_design_policy.pdf
pdf/general_document_changeset_policy.pdf
pdf/metrics.pdf
pdf/proposed_ontology.pdf
pdf/reports.pdf
pdf/requirements_policy.pdf
pdf/software_requirements_specification.pdf
pdf/testing_policy.pdf
pdf/wiki_usage_policy.pdf
\end{verbatim}
The file of particular interest right now is
\texttt{TracRequirements-0.1-py2.4.egg}, which is known as the `egg'. Note that the
string \texttt{py2.4} may differ depending on your Python version (see section
\ref{sec:python-version} above). Copy the egg to your Trac installation's
\texttt{plugins} directory. For example,
\begin{verbatim}
cp TracRequirements-0.1-py2.4.egg /usr/share/trac/plugins
\end{verbatim}
Next restart your Trac webserver.

After your Trac server has restarted, the Lifecycle Manager plugin should be
loaded. Check the `About/Plugins' page in your Trac installation at this
address: \texttt{http://trac-install/about/plugins}. In the list you should be
able to find `RequirementComponent,' `NewrequirementModule,' and
`RequirementModule.'

Next you are ready to set up permissions to use Lifecycle Manager.


\section{Permissions}

Lifecycle Manager has the following permission hierarchy:
\begin{itemize}
  \item \texttt{REQUIREMENT\_APPEND} -- comment on a requirement
  \item \texttt{REQUIREMENT\_CREATE} -- create or modify a requirement
  	\begin{itemize}
      \item \texttt{REQUIREMENT\_MODIFY} -- modify the description of a
      requirement or comment on a requirement
    \end{itemize}
  \item \texttt{REQUIREMENT\_VIEW} -- view a requirement or list of requirements
  \item \texttt{REQUIREMENT\_MODIFY} -- modify the description of a requirement
  or comment on a requirement
  	\begin{itemize}
      \item \texttt{REQUIREMENT\_APPEND} -- comment on a requirement
    \end{itemize}
  \item \texttt{REQUIREMENT\_ADMIN} -- perform any action on a requirement
  	\begin{itemize}
      \item \texttt{REQUIREMENT\_CREATE} -- create or modify a requirement
      \item \texttt{REQUIREMENT\_VIEW} -- view a requirement or a list of requirements
    \end{itemize}
\end{itemize}

Your Trac users' permissions must be enabled to one or more of the above in
order to use Lifecycle Manager. For example, add the permission
\texttt{REQUIREMENT\_VIEW} to a user `joe', perform the following step:
\begin{verbatim}
trac-admin /path-to-trac-env permission add joe REQUIREMENT_VIEW
\end{verbatim}

\section{Conclusion}

After the egg has been installed and permissions delegated, Lifecycle Manager is
ready to use. Users who can view the list of requirements (permission
\texttt{REQUIREMENT\_VIEW}) will see a button in their Trac toolbar called
`Requirements.' Click this button to begin!

\end{document}
